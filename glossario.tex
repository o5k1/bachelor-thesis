
%**************************************************************
% Acronimi
%**************************************************************
\renewcommand{\acronymname}{Acronimi e abbreviazioni}

\newacronym[description={\glslink{apig}{Application Programming Interface}}]
    {api}{API}{Application Programming Interface}

\newacronym[description={\glslink{umlg}{Unified Modeling Language}}]
    {uml}{UML}{Unified Modeling Language}
		
\newacronym[description={\glslink{b2bg}{Business-to-Business}}]
    {b2b}{B2B}{Business-to-Business}
		
\newacronym[description={\glslink{iotg}{Intenet of Things}}]
    {iot}{IoT}{Intenet of Things}

%**************************************************************
% Glossario
%**************************************************************
%\renewcommand{\glossaryname}{Glossario}

\newglossaryentry{apig}
{
    name=\glslink{api}{API},
    text=Application Programming Interface,
    sort=api,
    description={in informatica con il termine \emph{Application Programming Interface API} (ing. interfaccia di programmazione di un'applicazione) si indica ogni insieme di procedure disponibili al programmatore, di solito raggruppate a formare un set di strumenti specifici per l'espletamento di un determinato compito all'interno di un certo programma. La finalità è ottenere un'astrazione, di solito tra l'hardware e il programmatore o tra software a basso e quello ad alto livello semplificando così il lavoro di programmazione}
}

\newglossaryentry{umlg}
{
    name=\glslink{uml}{UML},
    text=UML,
    sort=uml,
    description={in ingegneria del software \emph{UML, Unified Modeling Language} (ing. linguaggio di modellazione unificato) è un linguaggio di modellazione e specifica basato sul paradigma object-oriented. L'\emph{UML} svolge un'importantissima funzione di ``lingua franca'' nella comunità della progettazione e programmazione a oggetti. Gran parte della letteratura di settore usa tale linguaggio per descrivere soluzioni analitiche e progettuali in modo sintetico e comprensibile a un vasto pubblico}
}

\newglossaryentry{startupg}
{
    name=Startup,
    text=startup,
    sort=startup,
    description={in economia, con questo termine, si indica una nuova impresa nelle forme di un'organizzazione temporanea o una società di capitali in cerca di un business model ripetibile e scalabile.
La scalabilità è un elemento cardine di questa tipologia di impresa. L'avvio di un'attività imprenditoriale non scalabile, come l'apertura di un ristorante, non coincide dunque con la creazione di una startup ma, piuttosto, di una società tradizionale}
}

\newglossaryentry{b2bg}
{
    name=B2B,
    text=B2B,
    sort=Business-to-Business,
    description={ acronimo di “\emph{business-to-business}”. Locuzione utilizzata per descrivere le transazioni commerciali elettroniche tra imprese, distinguendole da quelle che intercorrono tra le imprese e altri gruppi, come quelle tra una ditta e i clienti individuali (B2C, acronimo di “\emph{business-to-consumer}”) oppure quelle tra una impresa e il governo (B2G, acronimo di “\emph{business-to-government}”)}
}

\newglossaryentry{earlystageg}
{
    name=Finanziamento early stage,
    text=finanziamento early stage,
    sort=finanziamento,
    description={ investimento atto a sostenere le fasi iniziali di un’azienda o di un business}
}

\newglossaryentry{iotg}
{
    name=IoT, %voce glossario
    text=IoT, %sostituisce parola nel testo
    sort=Internet of Things,
    description={acronimo di “\emph{internet of things}”. Neologismo riferito ad una evoluzione dell'uso della rete: gli oggetti si rendono riconoscibili e reagiscono di conseguenza grazie al fatto di poter comunicare dati su se stessi e accedere ad informazioni aggregate da parte di altri. Le sveglie suonano prima in caso di traffico, le scarpe da ginnastica trasmettono tempi, velocità e distanza per gareggiare in tempo reale con persone dall'altra parte del globo, i contenitori delle medicine avvisano i familiari se si dimentica di prendere il farmaco. 
Tutti gli oggetti possono acquisire un ruolo attivo grazie al collegamento alla Rete}
}

\newglossaryentry{mentoring}
{
    name=Mentoring, %voce glossario
    text=mentoring, %sostituisce parola nel testo
    sort=mentoring,
    description={metodologia di formazione che fa riferimento a una relazione uno a uno tra un soggetto con più esperienza (mentor) e uno con meno esperienza (\emph{junior}, \emph{mentee}, \emph{protégé}), cioè un allievo, al fine di far sviluppare a quest'ultimo competenze in ambito formativo, lavorativo e sociale e di sviluppare autostima, a livello educativo-scolastico}
}

\newglossaryentry{mentor}
{
    name=Mentor, %voce glossario
    text=mentor, %sostituisce parola nel testo
    sort=mentor,
    description={nell’ambito del mentoring, è il soggetto con più esperienza. Deve avere capacità relazionali, saper condurre colloqui e porre domande sagge, deve saper gestire le fasi del processo di mentoring}
}

\newglossaryentry{internationalschool}
{
    name=International School, %voce glossario
    text=International School, %sostituisce parola nel testo
    sort=international school,
    description={prevede un percorso didattico che si articola seguendo il modello anglosassone ed è certificato dall' \emph{International Baccalaureate Organization}. 
Al termine del percorso, con il conseguimento dell'\emph{IB-Diploma Programme} gli studenti ottengono un diploma equipollente alla Maturità italiana con un anno di anticipo rispetto al percorso italiano. Hanno accesso a tutte le Università italiane e ai più prestigiosi atenei del mondo}
}

\newglossaryentry{arg}
{
    name=Alternate Reality Game, %voce glossario
    text=Alternate Reality Game, %sostituisce parola nel testo
    sort=Alternate Reality Game,
    description={spesso riferito anche con l’acronimo “ARG”, è una situazione di gioco che collega internet al mondo reale. Solitamente si sviluppa attraverso numerosi strumenti web (blog, email, mini-siti) e presenta al giocatore una storia misteriosa con indizi che puntano al mondo reale (per esempio a monumenti o a veri e propri oggetti nascosti in determinate località)}
}

\newglossaryentry{cms}
{
    name=CMS, %voce glossario
    text=CMS, %sostituisce parola nel testo
    sort=CMS,
    description={acronimo di “\emph{Content Management System}”. Con questo termine si indica uno strumento software, installato su un server web, il cui obiettivo è facilitare la gestione dei contenuti di siti web, soprattutto a coloro che non hanno conoscenze specifiche sulla programmazione web}
}

\newglossaryentry{frontend}
{
    name=Front end, %voce glossario
    text=front end, %sostituisce parola nel testo
    sort=front end,
    description={termine che denota la parte, di una piattaforma web, visibile all'utente e con la quale esso può interagire. Ci si riferisce a questo concetto anche con il termine “\emph{interfaccia utente}”}
}

\newglossaryentry{backend}
{
    name=Back end, %voce glossario
    text=back end, %sostituisce parola nel testo
    sort=back end,
    description={termine che denota l'insieme delle applicazioni, relative ad una piattaforma web, con le quali l'utente non interagisce direttamente ma che sono essenziali al funzionamento del sistema}
}

\newglossaryentry{orm}
{
    name=ORM, %voce glossario
    text=ORM, %sostituisce parola nel testo
    sort=ORM,
    description={acronimo di “\emph{Object-Relational Mapping}”. Con questo termine ci si riferisce ad una tecnica di programmazione che favorisce l'integrazione tra sistemi software orientati agli oggetti e sistemi \glslink{rdbms}. Più nello specifico, un prodotto ORM fornisce accesso alla persistenza dei dati attraverso un'interfaccia ad oggetti, astraendo così dai dettagli dello specifico \glslink{rdbms}{RDBMS} utilizzato}
}

\newglossaryentry{rdbms}
{
    name=RDBMS, %voce glossario
    text=RDBMS, %sostituisce parola nel testo
    sort=RDBMS,
    description={acronimo di “\emph{Relational Database Management System}”. Con questo termine ci si riferisce ad un sistema per la gestione di basi di dati relazionali}
}

\newglossaryentry{snakecase}
{
    name=Snake case, %voce glossario
    text=snake case, %sostituisce parola nel testo
    sort=snake case,
    description={la notazione snake case è la pratica di scrivere parole composte o frasi separandone le parti tramite un trattino basso (\_), solitamente lasciandone tutti i caratteri in minuscolo}
}

\newglossaryentry{ridondanza}
{
    name=Ridondanza, %voce glossario
    text=ridondanze, %sostituisce parola nel testo
    sort=ridondanza,
    description={nell'ambito dei database, un dato è ridondante se può essere derivato da altri dati}
}

\newglossaryentry{json}
{
    name=JSON, %voce glossario
    text=JSON, %sostituisce parola nel testo
    sort=JSON,
    description={acronimo di “\emph{JavaScript Object Notation}”, è un formato adatto all'interscambio di dati fra applicazioni client-server}
}

\newglossaryentry{trait}
{
    name=Trait, %voce glossario
    text=Trait, %sostituisce parola nel testo
    sort=trait,
    description={costrutto del linguaggio PHP. I \emph{Trait} sono pensati per ridurre alcuni limiti imposti dall'assenza della derivazione multipla in PHP, permettendo allo sviluppatore di riutilizzare un insieme di metodi. Un Trait è simile ad una classe ma deve essere usato solo per raggruppare delle funzionalità, non è consentito istanziare un Trait al di fuori di una classe }
}

\newglossaryentry{mysql}
{
    name=MySQL, %voce glossario
    text=MySQL, %sostituisce parola nel testo
    sort=mysql,
    description={\glslink{rdbms}{RDBMS} open source, rappresenta una delle tecnologie più diffuse nel mondo dell'IT. Nasce nel 1996 per opera dell’azienda svedese Tcx e viene distribuito liberamente per favorirne la crescita}
}

\newglossaryentry{crud}
{
    name=CRUD, %voce glossario
    text=CRUD, %sostituisce parola nel testo
    sort=CRUD,
    description={acronimo di “\emph{Create, Read, Update, Delete}”. Riassume le operazioni di base che possono essere effettuate su un sistema di persistenza dati: creazione di una risorsa, visualizzazione di una risorsa, aggiornamento di una risorsa e rimozione di una risorsa}
}

\newglossaryentry{http}
{
    name=HTTP, %voce glossario
    text=HTTP, %sostituisce parola nel testo
    sort=HTTP,
    description={acronimo di “\emph{HyperText Transfer Protocol}”. Protocollo a livello applicativo usato come principale sistema per la trasmissione d'informazioni sul web}
}

\newglossaryentry{url}
{
    name=URL, %voce glossario
    text=URL, %sostituisce parola nel testo
    sort=URL,
    description={acronimo di “\emph{Uniform Resource Locator}”. Nella terminologia delle telecomunicazioni e dell'informatica, è una sequenza di caratteri che identifica univocamente l'indirizzo di una risorsa in Internet, tipicamente presente su un host server, come ad esempio un documento, un'immagine, un video, rendendola accessibile ad un client che ne faccia richiesta attraverso l'utilizzo di un web browser}
}







