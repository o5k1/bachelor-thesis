% !TEX encoding = UTF-8
% !TEX TS-program = pdflatex
% !TEX root = ../tesi.tex
% !TEX spellcheck = it-IT

%**************************************************************
\chapter{Conclusioni}
\label{cap:conclusioni}
%**************************************************************
\section{Raggiungimento degli obiettivi}
Il progetto di stage prevedeva i seguenti obiettivi:
\begin{enumerate}
	\item Refactoring del database che consente la persistenza dei dati utili alla piattaforma Plot;
	\item Creazione di un set di API REST che permettano di interagire con il database;
	\item Refactoring del \gls{frontend}\glsfirstoccur{} relativo al \gls{cms}\glsfirstoccur.
\end{enumerate}

A fronte del lavoro svolto e vista l'approvazione del tutor aziendale, tali obiettivi possono considerarsi raggiunti.\\
Il \gls{cms}\glsfirstoccur{} può essere utilizzato dall'azienda nell'ambito del progetto Plot e, in futuro, potrà facilmente essere esteso o modificato utilizzando le API messe a disposizione. \\
Per favorire la manutenzione del software è stata prodotta e consegnata all'azienda la documentazione del codice relativo a tutte le classi implementate e previste dalla progettazione.

%**************************************************************
\section{Bilancio formativo}
Durante il corso di studi sono stati introdotti e studiati i processi che scandiscono la vita di un'azienda informatica, l'attività di stage mi permesso innanzitutto di comprendere maggiormente l'utilità di tali processi. \\
Ho avuto inoltre la possibilità di approfondire il concetto di \gls{startupg}\glsfirstoccur{}: un modello d'impresa che mi ha sempre affascinato e che oggi risulta molto popolare. \\
Ho trovato molto utile, infine, poter conoscere nuove tecnologie e poter consolidare quelle da me già acquisite.
